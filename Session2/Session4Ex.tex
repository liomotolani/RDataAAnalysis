\documentclass[]{article}
\usepackage{lmodern}
\usepackage{amssymb,amsmath}
\usepackage{ifxetex,ifluatex}
\usepackage{fixltx2e} % provides \textsubscript
\ifnum 0\ifxetex 1\fi\ifluatex 1\fi=0 % if pdftex
  \usepackage[T1]{fontenc}
  \usepackage[utf8]{inputenc}
\else % if luatex or xelatex
  \ifxetex
    \usepackage{mathspec}
  \else
    \usepackage{fontspec}
  \fi
  \defaultfontfeatures{Ligatures=TeX,Scale=MatchLowercase}
\fi
% use upquote if available, for straight quotes in verbatim environments
\IfFileExists{upquote.sty}{\usepackage{upquote}}{}
% use microtype if available
\IfFileExists{microtype.sty}{%
\usepackage{microtype}
\UseMicrotypeSet[protrusion]{basicmath} % disable protrusion for tt fonts
}{}
\usepackage[margin=1in]{geometry}
\usepackage{hyperref}
\hypersetup{unicode=true,
            pdftitle={Session4},
            pdfauthor={Omotolani Ligali},
            pdfborder={0 0 0},
            breaklinks=true}
\urlstyle{same}  % don't use monospace font for urls
\usepackage{color}
\usepackage{fancyvrb}
\newcommand{\VerbBar}{|}
\newcommand{\VERB}{\Verb[commandchars=\\\{\}]}
\DefineVerbatimEnvironment{Highlighting}{Verbatim}{commandchars=\\\{\}}
% Add ',fontsize=\small' for more characters per line
\usepackage{framed}
\definecolor{shadecolor}{RGB}{248,248,248}
\newenvironment{Shaded}{\begin{snugshade}}{\end{snugshade}}
\newcommand{\AlertTok}[1]{\textcolor[rgb]{0.94,0.16,0.16}{#1}}
\newcommand{\AnnotationTok}[1]{\textcolor[rgb]{0.56,0.35,0.01}{\textbf{\textit{#1}}}}
\newcommand{\AttributeTok}[1]{\textcolor[rgb]{0.77,0.63,0.00}{#1}}
\newcommand{\BaseNTok}[1]{\textcolor[rgb]{0.00,0.00,0.81}{#1}}
\newcommand{\BuiltInTok}[1]{#1}
\newcommand{\CharTok}[1]{\textcolor[rgb]{0.31,0.60,0.02}{#1}}
\newcommand{\CommentTok}[1]{\textcolor[rgb]{0.56,0.35,0.01}{\textit{#1}}}
\newcommand{\CommentVarTok}[1]{\textcolor[rgb]{0.56,0.35,0.01}{\textbf{\textit{#1}}}}
\newcommand{\ConstantTok}[1]{\textcolor[rgb]{0.00,0.00,0.00}{#1}}
\newcommand{\ControlFlowTok}[1]{\textcolor[rgb]{0.13,0.29,0.53}{\textbf{#1}}}
\newcommand{\DataTypeTok}[1]{\textcolor[rgb]{0.13,0.29,0.53}{#1}}
\newcommand{\DecValTok}[1]{\textcolor[rgb]{0.00,0.00,0.81}{#1}}
\newcommand{\DocumentationTok}[1]{\textcolor[rgb]{0.56,0.35,0.01}{\textbf{\textit{#1}}}}
\newcommand{\ErrorTok}[1]{\textcolor[rgb]{0.64,0.00,0.00}{\textbf{#1}}}
\newcommand{\ExtensionTok}[1]{#1}
\newcommand{\FloatTok}[1]{\textcolor[rgb]{0.00,0.00,0.81}{#1}}
\newcommand{\FunctionTok}[1]{\textcolor[rgb]{0.00,0.00,0.00}{#1}}
\newcommand{\ImportTok}[1]{#1}
\newcommand{\InformationTok}[1]{\textcolor[rgb]{0.56,0.35,0.01}{\textbf{\textit{#1}}}}
\newcommand{\KeywordTok}[1]{\textcolor[rgb]{0.13,0.29,0.53}{\textbf{#1}}}
\newcommand{\NormalTok}[1]{#1}
\newcommand{\OperatorTok}[1]{\textcolor[rgb]{0.81,0.36,0.00}{\textbf{#1}}}
\newcommand{\OtherTok}[1]{\textcolor[rgb]{0.56,0.35,0.01}{#1}}
\newcommand{\PreprocessorTok}[1]{\textcolor[rgb]{0.56,0.35,0.01}{\textit{#1}}}
\newcommand{\RegionMarkerTok}[1]{#1}
\newcommand{\SpecialCharTok}[1]{\textcolor[rgb]{0.00,0.00,0.00}{#1}}
\newcommand{\SpecialStringTok}[1]{\textcolor[rgb]{0.31,0.60,0.02}{#1}}
\newcommand{\StringTok}[1]{\textcolor[rgb]{0.31,0.60,0.02}{#1}}
\newcommand{\VariableTok}[1]{\textcolor[rgb]{0.00,0.00,0.00}{#1}}
\newcommand{\VerbatimStringTok}[1]{\textcolor[rgb]{0.31,0.60,0.02}{#1}}
\newcommand{\WarningTok}[1]{\textcolor[rgb]{0.56,0.35,0.01}{\textbf{\textit{#1}}}}
\usepackage{graphicx,grffile}
\makeatletter
\def\maxwidth{\ifdim\Gin@nat@width>\linewidth\linewidth\else\Gin@nat@width\fi}
\def\maxheight{\ifdim\Gin@nat@height>\textheight\textheight\else\Gin@nat@height\fi}
\makeatother
% Scale images if necessary, so that they will not overflow the page
% margins by default, and it is still possible to overwrite the defaults
% using explicit options in \includegraphics[width, height, ...]{}
\setkeys{Gin}{width=\maxwidth,height=\maxheight,keepaspectratio}
\IfFileExists{parskip.sty}{%
\usepackage{parskip}
}{% else
\setlength{\parindent}{0pt}
\setlength{\parskip}{6pt plus 2pt minus 1pt}
}
\setlength{\emergencystretch}{3em}  % prevent overfull lines
\providecommand{\tightlist}{%
  \setlength{\itemsep}{0pt}\setlength{\parskip}{0pt}}
\setcounter{secnumdepth}{0}
% Redefines (sub)paragraphs to behave more like sections
\ifx\paragraph\undefined\else
\let\oldparagraph\paragraph
\renewcommand{\paragraph}[1]{\oldparagraph{#1}\mbox{}}
\fi
\ifx\subparagraph\undefined\else
\let\oldsubparagraph\subparagraph
\renewcommand{\subparagraph}[1]{\oldsubparagraph{#1}\mbox{}}
\fi

%%% Use protect on footnotes to avoid problems with footnotes in titles
\let\rmarkdownfootnote\footnote%
\def\footnote{\protect\rmarkdownfootnote}

%%% Change title format to be more compact
\usepackage{titling}

% Create subtitle command for use in maketitle
\providecommand{\subtitle}[1]{
  \posttitle{
    \begin{center}\large#1\end{center}
    }
}

\setlength{\droptitle}{-2em}

  \title{Session4}
    \pretitle{\vspace{\droptitle}\centering\huge}
  \posttitle{\par}
    \author{Omotolani Ligali}
    \preauthor{\centering\large\emph}
  \postauthor{\par}
      \predate{\centering\large\emph}
  \postdate{\par}
    \date{10/31/2019}


\begin{document}
\maketitle

\#QUESTION 1: One sample t-test

\#Answer:

\#Hypothesis

\#Null Hypothesis: The mean is equal to 0.4

\#Alternate Hypothesis: The mean is not equal to 0.4

\#Level of Significance: The level of significance is 0.05

\#Test:

\begin{Shaded}
\begin{Highlighting}[]
\NormalTok{dt<-}\KeywordTok{read.csv}\NormalTok{(}\StringTok{"data/diamond.csv"}\NormalTok{,}\DataTypeTok{header =}\NormalTok{ T)}
\NormalTok{carat<-}\KeywordTok{t.test}\NormalTok{(dt}\OperatorTok{$}\NormalTok{carat,}\DataTypeTok{mu =} \FloatTok{0.4}\NormalTok{,}\DataTypeTok{alternative =} \StringTok{"greater"}\NormalTok{)}
\NormalTok{carat}
\end{Highlighting}
\end{Shaded}

\begin{verbatim}
## 
##  One Sample t-test
## 
## data:  dt$carat
## t = 194.98, df = 53939, p-value < 2.2e-16
## alternative hypothesis: true mean is greater than 0.4
## 95 percent confidence interval:
##  0.7945826       Inf
## sample estimates:
## mean of x 
## 0.7979397
\end{verbatim}

\#Conclusion: The null hypothesis is rejected because the p-value from
the test is less than 0.05

\#QUESTION 2: Two sample independent T-test

\#Answer:

\#Hypothesis

\#Null Hypothesis: The difference in mean of group M and group B is
equal to zero

\#Alternate Hypothesis: The difference in mean of group M and group B is
not equal to zero

\#Level of Significance: The level of significance is 0.05

\#Test:

\begin{Shaded}
\begin{Highlighting}[]
\NormalTok{ct<-}\KeywordTok{read.csv}\NormalTok{(}\StringTok{"data/cancer.csv"}\NormalTok{,}\DataTypeTok{header =}\NormalTok{ T)}
\KeywordTok{t.test}\NormalTok{(ct}\OperatorTok{$}\NormalTok{area_worst}\OperatorTok{~}\NormalTok{ct}\OperatorTok{$}\NormalTok{diagnosis,}\DataTypeTok{alternative =}\StringTok{"two.sided"}\NormalTok{)}
\end{Highlighting}
\end{Shaded}

\begin{verbatim}
## 
##  Welch Two Sample t-test
## 
## data:  ct$area_worst by ct$diagnosis
## t = -20.571, df = 229.91, p-value < 2.2e-16
## alternative hypothesis: true difference in means is not equal to 0
## 95 percent confidence interval:
##  -946.0847 -780.6890
## sample estimates:
## mean in group B mean in group M 
##        558.8994       1422.2863
\end{verbatim}

\#Conclusion: The null hypothesis is rejected. Therefore the p-value
from the test is less than 0.05 and the true difference in mean is not
equal to zero

\#QUESTION 3:Two sample paired test

\#Answer:

\#Hypothesis

\#Null Hypothesis: The difference in mean of tumor size before treatment
and tumor size after treatment is equal to zero

\#Alternate Hypothesis: The difference in mean of tumor size before
treatment and tumor size after treatment is not equal to zero

\#Level of Significance: The level of significance is 0.05

\#Test:

\begin{Shaded}
\begin{Highlighting}[]
\NormalTok{tumor<-}\KeywordTok{c}\NormalTok{(}\DecValTok{80}\NormalTok{, }\DecValTok{45}\NormalTok{, }\DecValTok{78}\NormalTok{, }\DecValTok{89}\NormalTok{, }\DecValTok{80}\NormalTok{, }\DecValTok{90}\NormalTok{, }\DecValTok{91}\NormalTok{, }\DecValTok{89}\NormalTok{, }\DecValTok{56}\NormalTok{, }\DecValTok{78}\NormalTok{, }\DecValTok{89}\NormalTok{,}\DecValTok{81}\NormalTok{)}
\end{Highlighting}
\end{Shaded}

\begin{Shaded}
\begin{Highlighting}[]
\NormalTok{tumor1<-}\KeywordTok{c}\NormalTok{(}\DecValTok{61}\NormalTok{, }\DecValTok{71}\NormalTok{, }\DecValTok{62}\NormalTok{, }\DecValTok{78}\NormalTok{, }\DecValTok{88}\NormalTok{, }\DecValTok{71}\NormalTok{, }\DecValTok{78}\NormalTok{, }\DecValTok{79}\NormalTok{, }\DecValTok{67}\NormalTok{, }\DecValTok{66}\NormalTok{,}\DecValTok{78}\NormalTok{,}\DecValTok{79}\NormalTok{)}
\end{Highlighting}
\end{Shaded}

\begin{Shaded}
\begin{Highlighting}[]
\KeywordTok{t.test}\NormalTok{(tumor,tumor1,}\DataTypeTok{paired =} \OtherTok{TRUE}\NormalTok{)}
\end{Highlighting}
\end{Shaded}

\begin{verbatim}
## 
##  Paired t-test
## 
## data:  tumor and tumor1
## t = 1.4164, df = 11, p-value = 0.1843
## alternative hypothesis: true difference in means is not equal to 0
## 95 percent confidence interval:
##  -3.138663 14.471997
## sample estimates:
## mean of the differences 
##                5.666667
\end{verbatim}

\#Conclusion: The null hypothesis is rejected. Therefore the p-value
from the test is less than 0.05 and the true difference in mean is not
equal to zero


\end{document}
